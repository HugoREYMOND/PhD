\section{Bibliography}

\subsection{Powder production processes}

\subsubsection{Gas atomization}
\subsubsection{Water atomization}
\subsubsection{Centrifugal atomization}
\subsubsection{Plasma atomization}
\subsubsection{Mechanical attrition and alloying}
\subsubsection{Melt spinning}
\subsubsection{Rotating electrode process (REP)}
\subsubsection{Chemical processes}

\subsection{Factors influencing metallic powder size and quality during gas atomization}

\subsubsection{Feedstock melting}

\textcolor{orange}{Feedstock property and geometry (powder, scrap, wire, metal alloy elements or pre-alloyed) impact the powder quality. If possible, taking a sample of the homogenous melted material allowing to analyze the chemistry can help to decide whether an alloying element is needed to be added \cite{Kassym2020Atomizationprocessesmetal}}.\\

Both opened and closed melting systems are used in gas atomization. \\

\subsubsection{Gaz environment}

\subsubsection{Nozzle geometry}

\subsubsection{Thermal condition}

\subsection{Powder caracterization}

\subsubsection{Ductility and hardness}
\subsubsection{Impurities and reactivity}
\subsubsection{Tap density, apparent density, compressibility, green strength, flow properties and compressibility}
