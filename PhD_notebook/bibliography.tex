\section{Bibliography}

\subsection{Powder production processes}

\subsubsection{Gas atomization}
\subsubsection{Water atomization}
\subsubsection{Centrifugal atomization}
\subsubsection{Plasma atomization}
\subsubsection{Mechanical attrition and alloying}
\subsubsection{Melt spinning}
\subsubsection{Rotating electrode process (REP)}
\subsubsection{Chemical processes}

\subsection{Factors influencing metallic powder size and quality during gas atomization}

\subsubsection{Feedstock melting}

One of the melting process is to use a crucible heated by induction. It allows different type of feedstock as powder, scrap, wire, and can accept pure metal elements or pre-alloyed elements. All these parameters will impact the powder quality. For instance, the melted scraps can be heterogeneous and can come with oxidation and impurities. In these cases, it is recommended to take a sample of the homogenous melted material in order to analyze the chemistry \cite{Kassym2020Atomizationprocessesmetal}.\\

Both opened and closed melting systems are used in gas atomization. \\



\subsubsection{Gaz environment}

\subsubsection{Nozzle geometry}

\subsubsection{Thermal condition}

\subsection{Powder caracterization}

\subsubsection{Ductility and hardness}
\subsubsection{Impurities and reactivity}
\subsubsection{Tap density, apparent density, compressibility, green strength, flow properties and compressibility}
