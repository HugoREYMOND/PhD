%..........................DOCUMENT CLASS DEFINITION AND PACKAGES.............

\documentclass{article}
\usepackage[english]{babel}
\usepackage[latin1]{inputenc}
\usepackage{graphicx,wrapfig,color,caption}
\usepackage{subcaption}
\usepackage[bottom]{footmisc}
\usepackage{amsmath}
\usepackage[svgnames,usenames,dvipsnames,table]{xcolor}
\usepackage{booktabs}
\usepackage{lastpage}
\usepackage[includefoot]{geometry}
\usepackage{bm}
\usepackage{amsfonts}
\usepackage{upgreek}
\usepackage{fancyhdr}
\usepackage[absolute]{textpos}
\usepackage{multirow}
\usepackage{indentfirst}
\usepackage{vmargin}
\usepackage{float}
\usepackage[section]{placeins}
\usepackage{lastpage}
\usepackage{fontawesome5}
\usepackage{rotating} % Unable 90° rotated tables
\usepackage{enumitem} % Being able to ask No indent (\begin{itemize}[leftmargin=*]) for item (bulletpoint)
\usepackage{booktabs} % Table \midrule on specific columns: \cmidrule(lr){2-4}
\usepackage{makecell} % To break line into a table cell: \makecell{This is \\ a too long text}
\usepackage{titlesec} % for customizing section titles
%\usepackage{svg}
%\svgsetup{inkscapeexe="C:/Program Files/Inkscape/bin/inkscape.exe"}
\usepackage[pdfusetitle]{hyperref} % This one has to be the last usepackage in the list

%..........................SETTINGS: MARGINS.............

\setpapersize{A4}
\setmarginsrb
{2cm} % left margin
{1cm} % upper margin
{2cm} % right margin
{1cm} % lower margin
{1cm} % header size
{1cm} % distance between header and text
{1cm} % footer size
{1cm} % distance between text and footer


%..........................HYPERSETUP.............

\hypersetup{
	colorlinks=true,
	linkcolor=blue,
	citecolor=blue} 


%..........................SETTINGS: HEADER AND FOOTER.............

\pagestyle{fancy} %=user defined
\fancyhf{}

\lhead{}
\chead{}
\rhead{}
\lfoot{}
\cfoot{\thepage \hspace{1pt} / \pageref{LastPage}}
\rfoot{}


%..........................SETTINGS: TITLE NUMEROTATION AND TABLE OF CONTENT.............

\setcounter{tocdepth}{3} %Put in the table of content <= type 5
\setcounter{secnumdepth}{5} %Put a number before the title <= type 5

%..........................SETTINGS: CAPTIONS CENTERED.............

\captionsetup{justification=centering, font=it}

%..........................SETTINGS: TITLE COLORS.............

% Set default colors for sections
\titleformat{\section}
  {\color{DarkBlue}\normalfont\large\bfseries} % style of the title
  {\thesection}                         % numbering
  {2mm}                                    % spacing label - title
  {}                                       % code before the title
\titlespacing*{\section} % * = absolute indentation from margin
  {5mm}   % Indentation from left margin
  {5mm}    % Space above the title
  {4mm}    % Space below the title


\titleformat{\subsection}
  {\color{DarkRed}\normalfont\large\bfseries} % style of the title
  {\thesubsection}                         % numbering
  {2mm}                                    % spacing label - title
  {}                                       % code before the title
\titlespacing*{\subsection} % * = absolute indentation from margin
  {5mm}   % Indentation from left margin
  {2mm}    % Space above the title
  {2mm}    % Space below the title


\titleformat{\subsubsection}
  {\color{darkgray}\normalfont\large\bfseries} % style of the title
  {\thesubsubsection}                      % numbering
  {2mm}                                    % spacing label - title
  {}                                       % code before the title
  \titlespacing*{\subsubsection} % * = absolute indentation from margin
  {10mm}   % Indentation from left margin
  {2mm}    % Space above the title
  {2mm}    % Space below the title


% ...........................FIRST PAGE..................................

\begin{document}
	
	\title{Experimental Optimization of Gas Atomization for Additive Manufacturing}
	\author{Hugo REYMOND}
	\date{}  % <--- no date shown, comment the line to let the date appear
	
	
	\maketitle % Put "Title" and "author" in the first page properly
	
	
	
	%..........................TABLE OF CONTENTS / FIGURES / LIST OF TABLE ........................
	
	\newpage
	\tableofcontents\thispagestyle{fancy}
	%\newpage
	%\listoffigures\thispagestyle{fancy}
	%\newpage
	%\listoftables\thispagestyle{fancy}
	
	%----------------------------------------------------------------------------------------------
	
	\newpage

%__________________________________________________________________________________________________


\section{Factors influencing metallic powder size and quality during gas atomization for metal 3D printing}

During \textbf{gas atomization} used to produce metallic powders for \textbf{metal additive manufacturing} (Laser Powder Bed Fusion (LPBF), Electron Beam Melting (EBM), and Directed Energy Deposition (DED)), the \textbf{particle size} and \textbf{powder quality} (sphericity, cleanliness, particle size distribution, internal porosity) are directly influenced by several key factors. These factors can be grouped into \textit{melt properties}, \textit{process parameters}, and \textit{environmental conditions} \cite{Kassym2020Atomization,Ren2025PowderReview,Zhang2014GasAtomizationReview}.

\subsection{Properties of the molten metal}

\begin{itemize}
    \item Lower viscosity $\rightarrow$ easier jet breakup $\rightarrow$ finer particles \cite{Lefebvre1989Atomization,Zhang2014GasAtomizationReview}
    \item High surface tension $\rightarrow$ larger droplets \cite{Lefebvre1989Atomization}
    \item Higher density $\rightarrow$ more energy needed to fragment (generally produce coarser powder) \cite{Zhang2014GasAtomizationReview}
    \item Alloying elements affect viscosity and surface tension. Oxides and inclusions $\rightarrow$ sphericity degradation and satellite formation \cite{Beckers2020Morphology,Uhlenwinkel2014Porosity}.
\end{itemize}

\subsection{Gas atomization process parameters}
\begin{itemize}
  \item Higher gas pressure/velocity $\rightarrow$ finer particles \cite{Zhang2014GasAtomizationReview,Beckers2020Morphology}\\
  /!$\backslash$ Excessive values $\rightarrow$ satellites and irregular particles \cite{Beckers2020Morphology}
  \item \textbf{Helium}: very fine particles (low density, high velocity) \cite{Kassym2020Atomization,Uhlenwinkel2014Porosity}
  \item \textbf{Argon}: good quality (but cost compromise) \cite{Ren2025PowderReview}
  \item \textbf{Nitrogen}: economical but reactive with some alloys (Ti, Al) \cite{Ren2025PowderReview}
  \item High gaz-to-metal massflow ratio $\rightarrow$ improve fragmentation \cite{Cacace2024GasMetalRatio}\\
  /!$\backslash$ Excessive values $\rightarrow$ jet instability and broader particle size distrubition \cite{Cacace2024GasMetalRatio}
  \item \textbf{Hydrogen} (less common due to high risk of explosion): Very low molecular weight $\rightarrow$ high gas velocity $\rightarrow$ finer powder \cite{Kassym2020Atomization}\\
  /!$\backslash$ Soluble gases $\rightarrow$ internal porosity \cite{Uhlenwinkel2014Porosity}
\end{itemize}

\subsection{Nozzle geometry}
The nozzle design (gas jet angle, symmetry, gas-metal interaction distance) $\rightarrow$ affects particle size, sphericity, and powder yield \cite{Beckers2020Morphology,Zhang2014GasAtomizationReview}.

\subsection{Thermal conditions}
\begin{itemize}
    \item Higher superheat $\rightarrow$ lowers viscosity and improves atomization $\rightarrow$ finer and more spherical particles. Excessive values $\rightarrow$ evaporation of alloying elements, oxidation \cite{Zhang2014GasAtomizationReview,Ren2025PowderReview}
    \item Rapid solidification $\rightarrow$ spherical particles and fine microstructure. Slow cooling $\rightarrow$ deformed particles and satellite \cite{Uhlenwinkel2014Porosity,Beckers2020Morphology}\\
    /!$\backslash$ high cooling rates $\rightarrow$ increase interal porosity \cite{Uhlenwinkel2014Porosity}
\end{itemize}

\subsection{Atomization environment}
\begin{itemize}
    \item Oxygen and moisture $\rightarrow$ surface oxidation, reducing flowability, wettability, and laser absorptivity \cite{Ren2025PowderReview,Saheb2020ReviewPowders}
    \item reduces chamber pressure $\rightarrow$ improves fragmentation \cite{Zhang2014GasAtomizationReview}\\
    /!$\backslash$ High pressure increases droplet collisions and agglomeration \cite{Beckers2020Morphology}
\end{itemize}

\subsection{Summary of key influencing factors}

\begin{center}
\begin{tabular}{lcc}
\hline
\textbf{Factor} & \textbf{Particle Size} & \textbf{Powder Quality} \\
\hline
Gas pressure / velocity & Very high & High \cite{Zhang2014GasAtomizationReview} \\
Gas type & High & Very high \cite{Kassym2020Atomization} \\
Superheat temperature & High & High \cite{Ren2025PowderReview} \\
Nozzle geometry & Very high & Very high \cite{Beckers2020Morphology} \\
Atmosphere purity & Low & Very high \cite{Saheb2020ReviewPowders} \\
Metal properties & High & High \cite{Lefebvre1989Atomization} \\
\hline
\end{tabular}
\end{center}

\section{Powder quality caracterization}

\subsection{... of the particles, independantly}

\begin{itemize}
\item Size
\item Geometry (sphericity, satellite...)
\item Porosity
\item what else?
\end{itemize}

\subsection{... of the powder = behavior of the particles between each other}

\begin{itemize}
\item \textbf{Apparent density}: Mass of powder per unit volume in a loose, non-compacted state, including interparticle voids.
\item \textbf{Tap density}: Density of a powder after mechanical tapping or vibration, allowing particle rearrangement and packing.
\item \textbf{Compressibility}: Ability of a powder to decrease in volume under applied pressure.
\item \textbf{Flow properties}: Ability of a powder to flow consistently and uniformly under gravity or external forces.
\item \textbf{Green strength}: Mechanical strength of a compacted powder body before sintering.
\item What else?
\end{itemize}


\section{Powder quality impact on printed parts}

\subsection{Optimized powder parameters for LPBF}

LPBF (Laser Powder Bed Fusion) imposes strict requirements on powder characteristics to ensure stable recoating, uniform melting, and defect-free parts \cite{Saheb2020ReviewPowders,Ren2025PowderReview}.

\begin{itemize}
  \item Particle size distribution (PSD): \textbf{15--45 $\mu$m} (typical) \cite{Saheb2020ReviewPowders}
  \item D$_{10}$/D$_{50}$/D$_{90}$: narrow distribution preferred \cite{Ren2025PowderReview}
  \item Sphericity: $> 0.95$ \cite{Beckers2020Morphology}
  \item Apparent density: $> 50\%$ of theoretical density \cite{Saheb2020ReviewPowders}
  \item Hall flow rate: $< 25$ s / 50 g \cite{Saheb2020ReviewPowders}
  \item Oxygen content:
  \begin{itemize}
    \item Ti alloys: $< 0.15$ wt.\% \cite{Ren2025PowderReview}
    \item Al alloys: $< 0.10$ wt.\% \cite{Ren2025PowderReview}
    \item Steels/Ni alloys: $< 0.05$ wt.\% \cite{Ren2025PowderReview}
  \end{itemize}
\end{itemize}

\subsection{Typical powder defects and their impact on LPBF}

\begin{itemize}
  \item Satellite = droplet collision $\rightarrow$ poor powder flowability; irregular speading, increase porosity in printed part \cite{Beckers2020Morphology}
  \item Internal porosity (rapid solidification) $\rightarrow$ porisity transferted to printed part and reduce fatigue strength \cite{Uhlenwinkel2014Porosity}
  \item Wide Particle Size Distribution $\rightarrow$ Segregation during recoating, non-uniform melting behavior, surface roughness variation \cite{Saheb2020ReviewPowders}
\end{itemize}

\subsection{Powder quality-process-property relationship}

\begin{center}
\begin{tabular}{lcc}
\hline
\textbf{Powder Attribute}  & \textbf{Part Properties} \\
\hline
High sphericity & High density, smooth surface \cite{Beckers2020Morphology} \\
Low oxygen content & Improved ductility and fatigue \cite{Ren2025PowderReview} \\
Low internal porosity & High fatigue and fracture resistance \cite{Uhlenwinkel2014Porosity} \\
Narrow PSD & Dimensional accuracy \cite{Saheb2020ReviewPowders} \\
Good flowability & Reproducibility \cite{Saheb2020ReviewPowders} \\
\hline
\end{tabular}
\end{center}

\section{To dig}

\subsection{Powder manifactoring processes}

gas atomization, water atomization, centrifugal atomization,
plasma atomization, mechanical attrition and alloying, melt spinning, rotating electrode process (REP), and a variety of chemical
processes. (see ref [5] "Leo VM Antony, Ramana G. Reddy, Processes for production of high-purity metal powders, JOM 55 (3) (2003) 14-18." in the paper \cite{Kassym2020Atomization})

\subsection{Measurment methods}

For:
\begin{itemize}
    \item Size
    \item Geometry (sphericity, satellite)
    \item Porosity
    \item Apparent/Tap density, compressibility
    \item Flow properties
    \item Oxygen content
    \item \dots
\end{itemize}

\subsection{Coarsening}

Coarsening in metallurgy is a \textbf{thermally activated microstructural evolution} that occurs when a metal or alloy is exposed to \textbf{elevated temperature for a sufficient time}.  

In gas atomization of metal powders, classical coarsening is generally negligible due to extremely high cooling rates and short solidification times; it may only occur in large particles or during post-atomization thermal treatments.

\subsection{Liquidus / Superheat / Overheating}

Overheating brings oxidation even in inert gaz?

\section{To read}

\begin{itemize}
    \item Atomization processes of metal powders for 3D printing, Kassym, Kazybek and Perveen, Asma \cite{Kassym2020Atomization}
    \item Metal powder atomization preparation, modification, and reuse for additive manufacturing, Ren, P. and others \cite{Ren2025PowderReview}
    \item Impact of process flow conditions on particle morphology in metal powder production via gas atomization, eckers, D. and Ellendt, N. and Fritsching, U. and Uhlenwinkel, V., \cite{Beckers2020Morphology}
    \item Investigation on the effect of the gas-to-metal ratio on powder properties and PBF-LB/M processability, Cacace, S. and others \cite{Cacace2024GasMetalRatio}
    \item Gas atomization of duplex stainless steel powder for laser additive manufacturing, Cui, C. and others \cite{Cui2023DuplexSteel}
    \item A review on metal powders in additive manufacturing, Saheb, S. H. and others \cite{Saheb2020ReviewPowders}
    \item Review of gas atomisation and spray forming phenomenology, Zhang, R. and Zhang, Z. and Liu, Q., \cite{Zhang2014GasAtomizationReview}
    \item Atomization and Sprays, Lefebvre, Arthur H., \cite{Lefebvre1989Atomization}
    \item Gas atomization of metals, Uhlenwinkel, V. \cite{Uhlenwinkel2014Porosity}
    \item Numerical analysis of droplet breakup, cooling, and solidification during gas atomisation, Wang, Gezhou and Deng, Yuanbin and Adjei-Kyeremeh, Frank and Zhang, Jiali and Raffeis, Iris and Buhrig-Polaczek, Andreas and Kaletsch, Anke and Broeckmann, Christoph, \cite{wang_numerical_2023}
    \item Pre-breakup mechanism of free-fall nozzle in electrode induction melting gas atomization, Zou, Haiping and Xiao, Zhiyu, \cite{zou_pre-breakup_2021}
    \item Numerical simulation study on cooling of metal droplet in atomizing gas, Zhang, Min and Zhang, Zhaoming, \cite{zhang_numerical_2020}
    \item Effects of different nozzle materials on atomization results via {CFD} simulation, Li, Xiangyu and Du, Jianjun and Wang, Licheng and Fan, Jiangli and Peng, Xiaojun \cite{li_effects_2020}
\end{itemize}

\section{QUESTIONS}

\begin{enumerate}
    \item Each parameter might vary depending the metal material and the choosen inert gaz. In this PhD, do we want to optimize a specific material in a specific environement (at least as a stating point)? If yes, which ones?
    \item Plasma atomization produces higher-quality powders than gas atomization but comes at a high cost. Is one of the goals of this PhD to challenge plasma atomization by developing a cheaper alternative?

    \vspace{0mm}
    \begin{table}[H]
        \centering
       \begin{tabular}{l c c}
\toprule
        \textbf{Criterion}    &   \textbf{Gaz atomization}     &      \textbf{Plasma atomization} \\                
\midrule
            Production cost       & Moderate  &   High   \\
            Particule sphericity       & High  &   Very high   \\
            PSD       & Wide range (5-200 $\upmu$m)  &   Narrow range (10-60 $\upmu$m)   \\
            Satellite & Possible & Very rare \\
            Internal porosity & Possible & Very rare\\
            Material flexibility & Broad (steel, Al, Ni, Co) & Mainly Ti, Ni, reactive alloy\\
\bottomrule
                        \end{tabular}
        \caption{This is the table 1}
        \label{}
    \end{table}
\FloatBarrier
\vspace{0mm}
    
    \item During this PhD, the student has to teach during few hours or it is not mandatory?
    


\end{enumerate}

%\nocite{*}
% --> all the references will appear in the document
\bibliographystyle{unsrt}
% --> numbering in the order defined by the text.
% \renewcommand{\refname}{} % --> to hide the automatic "reference title"
\bibliography{bib.bib}

%__________________________________________________________________________________________________

\end{document}

1_ _ _ _ _ _ _ _ _ _ _ _ _ _ _ _ _ _ _ _ _ _ _ _ _
TITLES

\section{Title type 1}
\label{}
        \subsection{Title type 2}
        \label{}
                \subsubsection{Title type 3}
                \label{}
                        \paragraph{Title type 4\\[0.4\baselineskip]}
                        \label{}
                                \subparagraph{Title type 5\\[0.4\baselineskip]}
                                \label{}


_ _ _ _ _ _ _ _ _ _ _ _ _ _ _ _ _ _ _ _ _ _ _ _ _
TEXT SIZE

\Huge        
\huge        
\LARGE
\Large
\large        
\normalsize
\small
\footnotesize
\scriptsize
\tiny


_ _ _ _ _ _ _ _ _ _ _ _ _ _ _ _ _ _ _ _ _ _ _ _ _
TEXT COLOR

{\color{White} ....text....}
{\color{Grey} ....text....}
{\color{Black} ....text....}
{\color{Red} ....text....}
{\color{Orange} ....text....}
{\color{Yellow} ....text....}
{\color{ForestGreen} ....text....}
{\color{Green} ....text....}
{\color{Blue} ....text....}
{\color{Brown} ....text....}
{\color{Purple} ....text....}
{\color{Cyan} ....text....}
{\color{Magenta} ....text....}


_ _ _ _ _ _ _ _ _ _ _ _ _ _ _ _ _ _ _ _ _ _ _ _ _
ITEMS

-->     -
        -
        -
\begin{minipage}{\linewidth}
\begin{itemize}
    \item My text 1
    \item My text 2
    \item My text 3
\end{itemize}
\end{minipage}

-->     1)
        2)
        3)
        
\begin{minipage}{\linewidth}
\begin{enumerate}
    \item My text 1
    \item My text 2
    \item My text 3
\end{enumerate}
\end{minipage}

-->   [] --> (Empty) or [customized]

\begin{minipage}{\linewidth}
\begin{itemize}
                \item[] My text 1
                \item[] My text 2
                \item[] My test 3
\end{itemize}
\end{minipage}

_ _ _ _ _ _ _ _ _ _ _ _ _ _ _ _ _ _ _ _ _ _ _ _ _
INSERT IMAGE

\vspace{0mm}
\begin{figure}[H]
    \centering
        \includegraphics[width=9cm]{./img/My_picture.png}
                \caption{This is the caption of the figure 1}
                \label{}
\end{figure}
\FloatBarrier
\vspace{0mm}

\vspace{0mm}
\begin{figure}[H]
    \centering
        \includesvg[width=9cm]{./img/My_picture}
                \caption{This is the caption of the figure 1}
                \label{}
\end{figure}
\FloatBarrier
\vspace{0mm}


_ _ _ _ _ _ _ _ _ _ _ _ _ _ _ _ _ _ _ _ _ _ _ _ _
AGES SIDE-BY-SIDE

\begin{figure}
%
                \centering
                \begin{subfigure}[b]{0.5\textwidth}
                                \centering
                                  \includegraphics[width=\textwidth]{./img/My_picture.png}
                                \caption{This is the subcaption 1}
                                \label{}
                \end{subfigure}
\hfill
                \begin{subfigure}[b]{0.4\textwidth}
                   \begin{table}[H]
                                  \centering
                                      \begin{tabular}{c}
                        \includegraphics[width=\textwidth]{./img/My_picture.png}
                                      \end{tabular}
                                  \caption{This is the subcaption 2}
                                  \label{}
                   \end{table}
        \end{subfigure}
%
    \caption{This is the caption of the whole figure}
\end{figure}


_ _ _ _ _ _ _ _ _ _ _ _ _ _ _ _ _ _ _ _ _ _ _ _ _
INSERT IMAGES WITH TEXT ARROUND

\begin{wrapfigure}{r}{8cm}
                \centering\includegraphics[width=6cm]{./img/My_image.png}
                \caption{My caption}
                \label{}
\end{wrapfigure}


_ _ _ _ _ _ _ _ _ _ _ _ _ _ _ _ _ _ _ _ _ _ _ _ _
INSERT TABLE

\vspace{0mm}
    \begin{table}[H]
        \centering
       \begin{tabular}{|c|c|c|c|}
\hline
            \textbf{??}     &      \textbf{??}     &      \textbf{??}          &             \textbf{??}          \\                
            \hline
            ??       & ??  &   ??  &   ??  \\
            \hline
            ??       & ??  &   ??  &   ??  \\
            \hline
            ??       & ??  &   ??  &   ??  \\
              \hline
                        \end{tabular}
        \caption{This is the table 1}
        \label{}
    \end{table}
\FloatBarrier
\vspace{0mm}
                
\multirow{3}{*}{Text}
\multicolumn{2}{l (or) c (or) r}{Text}
\makecell[l]{Text too long on one row \\ next part of the text}
cell 1  &   \cellcolor{blue} My_text_cell_2 & cell 3


_ _ _ _ _ _ _ _ _ _                           
INSERT TABLE (paper-like)

\vspace{0mm}
    \begin{table}[H]
        \centering
       \begin{tabular}{c c c c}
\toprule
            \textbf{??}     &      \textbf{??}     &      \textbf{??}          &        \textbf{??}       \\                
\midrule
            ??       & ??  &   ??  &   ??  \\
            ??       & ??  &   ??  &   ??  \\
            ??       & ??  &   ??  &   ??  \\
\bottomrule
                        \end{tabular}
        \caption{This is the table 1}
        \label{}
    \end{table}
\FloatBarrier
\vspace{0mm}

\cmidrule(lr){2-4} %\midrule on specific columns


_ _ _ _ _ _ _ _ _ _                           
INSERT TABLE 90° ROTATED (paper-like)

\begin{sidewaystable}[ht]
\centering
\caption{Example Table in Vertical Orientation}
\begin{tabular}{lllll}
\toprule
Header 1 & Header 2 & Header 3 & Header 4 & Header 5 \\
\midrule
Data 1   & Data 2   & Data 3   & Data 4   & Data 5   \\
Data 6   & Data 7   & Data 8   & Data 9   & Data 10  \\
Data 11  & Data 12  & Data 13  & Data 14  & Data 15  \\
\bottomrule
\end{tabular}
\end{sidewaystable}

                
_ _ _ _ _ _ _ _ _ _ _ _ _ _ _ _ _ _ _ _ _ _ _ _ _
INSERT IMAGE --> TABLE

\begin{table}[H]
                \centering
                                \begin{tabular}{c}
                                                \includegraphics[width=17cm]{./img/My_table.png}
                                \end{tabular}
                \caption{My caption}
                \label{}
\end{table}
\FloatBarrier


_ _ _ _ _ _ _ _ _ _ _ _ _ _ _ _ _ _ _ _ _ _ _ _ _
INSERT MINIPAGE

\begin{minipage}[]{5cm}
    My text 1 or my picture 1
\end{minipage}
%
\begin{minipage}[]{12cm}
    My text 2 or my picture 2
\end{minipage}

%              [t] = top
%              [b] = bottom


_ _ _ _ _ _ _ _ _ _ _ _ _ _ _ _ _ _ _ _ _ _ _ _ _
REFERENCES IN THE TEXT

...in the picture \ref{Label of the the picture}
... in the chapter \ref{Label of the chapter} (page \pageref{Label of the chapter})
...\cite{The name of the bibliography}


_ _ _ _ _ _ _ _ _ _ _ _ _ _ _ _ _ _ _ _ _ _ _ _ _
FOOTNOTE

My text\footnote{My footnote}


_ _ _ _ _ _ _ _ _ _ _ _ _ _ _ _ _ _ _ _ _ _ _ _ _
CARACTERS AND MATHS

Lowercase letter $\gamma$
Uppercas letter $\Gamma$
Italic letter $\gamma$
Non-italic + lowercase letter $\upgamma$
Non-italic + uppercase letter $\Upgamma$


\begin{center}
    \begin{equation}
                    \boxed{Put your equation here}
                \end{equation}
\end{center}


\begin{align*} 
    A &=  1+1 \\ 
    A &=  2
\end{align*}


\begin{equation} 
    \begin{split}
        A & = A^2 \\
                    & = A.A
                \end{split}
\label{}
\end{equation}


\[
    Formula centered
\]

\[
    \boxed{Formula centered + framed}
\]


_ _ _ _ _ _ _ _ _ _ _ _ _ _ _ _ _ _ _ _ _ _ _ _ _
SETTINGS LATEX FOR INKSCAPE

Modify in options: 
--> Option/Configurer --> TeXstudio/Compilations

In PDFLaTeX, copy/paste this:
pdflatex.exe -synctex=1 -interaction=nonstopmode --shell-escape %.tex

Press OK

In your LaTeX script, add:
usepackage{svg}

... and also:
\svgsetup{inkscapeexe="C:/Program Files/Inkscape/bin/inkscape.exe"}    

(find the right location of "inkscape.exe" in your computer)

To import a figure, please use:

\begin{figure}[H]
    \centering
    \includesvg[width=.7\textwidth]{nom_fichier}
    \caption{ton_titre}
    \label{ton_label }
 \end{figure}

 (See that there is no ".svg")




_ _ _ _ _ _ _ _ _ _ _ _ _ _ _ _ _ _ _ _ _ _ _ _ _
BIBLIOGRAPHY

\nocite{*}
--> all the references will appear in the document
\bibliographystyle{unsrt} --> numbering in the order defined by the text.
% \renewcommand{\refname}{} % --> to hide the automatic "reference title"
\bibliography{My file name.bib}

example of "Bibligraphy.bib" content:

@article{Mecaniquedesstructures,
                        author = {S. Laroze and J. J. Barreau},
                        title = {Mecanique des structures},
                        journal = {Edition MASSON},
                        year = {1988},
                }





